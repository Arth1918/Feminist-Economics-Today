\documentclass[UTF8]{ctexart}

\title{护理“契约”}
\author{Paula England与Nancy Folbre}

\begin{document}
\maketitle

传统职责中,女性照顾受养人的无偿工作,导致了她们经济上依赖男性,并在劳动力市场中处于不利地位。
相较于其他女性,母亲们的终身薪资大幅减少,且在不婚或离婚的情况下面临着显著的贫困风险
(Budig和Engl\newline and,2001;Davies,Joshi和Peronaci,2000;Waldfogel。1998;Joshi,199\newline 0)。而更多的护理工作则成为有偿工作,
大多数由在要求同样教育和经验的其他工作中收入较低的女性包揽。(Budig,England和Folbre,2001;England和Folbre,1999)。
尽管承担护理工作的人大多数都收入很低,但其成本相对于其它商品和服务仍在增长,因此有时,需要护理服务的人却无法负担。
联邦政府正在考虑为在家照顾孩子的母亲的公共援助设置限制,并试图控制医疗成本,尤其是老年人的医疗成本。提供给受养人,如儿童和老年人的护理服务似乎不平衡,
而在某些情况下低得令人无法接受。简而言之,我们经济体中的“护理部门”存在许多严重问题。
什么导致了这些问题?许多经济学家认为,被女性承担的护理服务和其他工作的薪资过低,是因为在这些领域中劳动力的过度供应。
显性歧视和性别角色社会化都限制了女性的就业机会,并迫使他们从事传统的“女性工作”(Bergmann,1981,1986;Jacobsen,1994;Blau,Ferber和Winkler,1998)。
针对比较价值的研究表明,文化偏见也会产生影响,即雇主往往会贬低女性承担的工作的价值,并为其设置更低的的薪资标准(England,1992)。这些因素都助于说明了护理工作的回报过低的原因。
然而,依我们看来,它们并没有提供足够全面的解释,对上述其他问题也几乎没有说明。
在这篇文章当中,我们认为护理工作,其自身拥有可以帮助解释从业者的经济弱势的显著特征。
高质量的护理服务通常要求以情感联结,道德义务和内在动机为特征的长期责任或“契约”(“contracts”)。无论这样的“契约”表现为主要受社会规范制约的隐性共识,
或是雇主和雇员之间的显性协定,它们都难以明确并执行。然而,它们对儿童,病人和老人等受养人的福祉尤为重要,因为他们很少有能力进行协商。
我们的分析明确地将经济理论中的“新制度主义”方法与女性主义理论联系起来。大部分关于护理的跨学科女性主义文献主要从批评的角度对待经济理论。
(见,Himmelweit,1999;Kaber,2001;Steven;Held,2002)。这并不足为奇,因为大多数的经济学家假定个体们是“理性的”,自私自利的决策者们主要对价格和收入的变化做出反应。
大多数的经济学者们也同样关注无长期合同义务的,非个人市场当中的交易,然而护理工作具有重要的人文关怀和利他主义内涵,并受价值观和社会规范的影响(见England,本书第一章),
对此,很难找到比其更适合分析护理工作的理论了。护理工作天然带有个人性质,且存在于相对长期的关系中。
主流经济学在这样的问题中仍然留下了大片空白。但是,这种主流,正在愈发扩散开来,伴随着涌现出新的,
运用着“交易成本”、“隐形合约”、“外生性偏好”、“互惠”等概念
(Williamson 1985; Pollak 1985; Stiglitz 1987; Akerlof 1982; Bowles and Gintis 1998)的潜在支流。
简短地说,我们将这种文献视为“新制度主义经济学”,并以此大体上定义它。尽管其相当依赖于传统经济学假设的基础,新制度主义的方法总体上强调了价值观,规范与偏好影响个体决策协调的方式,
并常用“契约”一词作为比喻来解释非市场制度和长期关系的演化。尽管有其自身的局限性,我们仍然认同这样的比喻为社会护理组织提供了某些重要的经济视角。
我们也同样认同这种“契约”的话语提供了女性主义的担忧,有助于重新引导主流经济话语。
尽管“契约”意味着可能会限制未来选择的约束性承诺,但人们仍然持续地选择参与,或适应显性与隐性的契约。因此,这种“契约”的比喻似乎提供了一种既能保留个人选择的因素,
又能同时解释其限制因素的诱人思路。然而,有些契约,却相比其他契约更难以设计与执行。我们认为“护理契约”就容易被三个问题影响:
(1)人们不能完全参与到对他们有影响的契约制定当中;(2)“护理契约”难以监管与执行;(3)人们会被他们所参与的契约所改变。这些问题不仅导致了性别不平等,
更导致了有偿护理工作的薪资低、服务质量低以及高质量劳动力的供应不足。它们都体现出反思并重新设计社会护理组织的重要性。

\section{护理的女性主义概念化}
大量的女性主义文献都批评西方学术传统对护理的忽视,并强调其在女性生活中的中心性和其对整个社会的重要性。
这些文献的大多数都突出强调了护理作为活动的独特特征,即显然违背了根据“理性经济人”的动机——对狭隘私利的极致追求,所做出的假设(Staveren,2001)。
这意味着不存在“护理”与“不护理”的二元对立,而是存在一个连续体当中的特定位置。
护理往往具有尤其突出的情感色彩,且常常附带着强烈的道德义务。因此,它作为被高度性别化的概念,一个更倾向女性而非男性领域的概念(Nelson, 1996)。
“护理”这个词本身就常常被用于描述动机,或一种道德义务(Noddings 1984; Tronto 1987; Gordon, Benner, and Noddings 1996)。
基于同样的实质,一些社会科学家会采用“关怀劳动”来提醒人们护理意味着建立给予者与接收者之间的情感联结。
Kari Waerness(1987)与Arnaug Leira(1994)强调,护理工作如何脱离传统经济学对工作这一活动的定义,即尽管其内在效用不高,仅为了赚取金钱进行的活动。
Jean Gardiner(1997)同Sue Himmelweit(1999)都认为,将“工作”与家庭护理划上等号忽视了它的个人与情感意味,而有偿护理保持其私人质量则取决于其抵抗“完全商品化”的程度。
换句话说,护理的给予者并不纯粹受金钱奖励的驱使。Emily Abel与Margeret Nelson (1990, 4) 这样解释道:“给予护理是一项同时兼具工具性任务与情感性关系的活动。
尽管传统的帕森斯理论区别了这两种行为模式,护理的给予者们仍然期望提供爱,如同提供劳动一样。”
类似的,Francesca Cancian与Stacy Oliker (2000, 2) 将护理定义为“以面对面的方式,负责地为个人提供个人需求或福祉”的行动与感受的结合。
基于同样的精神,Nancy Folbre此前曾定义护理劳动为基于持续性的个人互动(通常是面对面的)提供服务,并(至少部分)受对接收者福利的关心所驱使。
(Folbre 1995; Folbre和Weisskopf 1998)
其他女性主义学者则就完成的任务和提供的服务来定义护理。因此,例如Mary Daly (2001)将护理定义为使受养人,例如病人,无法自理的老年人和儿童受益的所有活动;
Diemut Bubeck (1995, 183) 定义护理为“满足无法满足自身需求的人的需求”。Deborah Ward (1993) 将这个观念更广泛地应用于许多由市场之外的家庭与社群满足地个人需求。
尽管他们对结果有所强调,他们也同时强调性别化的社会规范在塑造提供护理的动机时的作用。
有经验的研究者们不能轻易地验证或测量动机,而因此他们更倾向于关注护理工作的其他特征。
在我们最近的研究中,我们将护理劳动定义为提供可以提高受养人能力的面对面服务 (England, Budig, 和Folbre 2001)。“能力”指健康,技能,或对个人自身或周围人有用的倾向。
这些包括生理与心理拮抗,以及物理技能,认知技能和情感技能,例如自律能力,共情能力,和关怀能力。这样的护理服务由父母,其他家庭成员,朋友或志愿者提供,但也包括以此得到报酬的人,
如教师,护士,儿童护理员,老人护理员,治疗师等等。我们发现受雇于这些工作,带来诸如薪资惩罚,教育网络,多年经验,和一定的工作特征,
例如性别结构,技能需求,产业,无论工人们是否组成工会,也无论他们为自雇或为政府工作,等等。
初步估计表明了以工作表现类型来定义的护理服务,



\end{document}